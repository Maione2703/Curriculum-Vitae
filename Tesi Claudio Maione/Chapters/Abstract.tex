\chapter*{Abstract}
\thispagestyle{chapterfancy}

%The main purpose of this Thesis is to verify the replicability of the results of Yueming Wu \textit{et al.} in their paper regarding Android Malware Classification \cite{wu2022contrastive} and expand their idea to a different environment: Linux IoT. \\
%First we will give a brief explanation of all the tools we need to perform this experiment, from Machine Learning to the Linux Kernel, and then we will discuss the Cybersecurity of Linux IoT, providing some insights about the specific Malware we will use in the experiment. \\
%To conclude, we will show the results of the tests based on Yueming Wu \textit{et al.} paper \cite{wu2022contrastive}, then we will discuss the needed modifications and the results of our interpretation of their idea applied to the Linux environment.

The main purpose of this thesis is the study and application of computer vision techniques based on Artificial Intelligence (AI) to malware classification. \\ 
We approach this topic by first replicating the results obtained by Yueming Wu \textit{et al.} in their paper regarding Android malware classification \cite{wu2022contrastive}. Afterwards, we extend those ideas to the Linux IoT environment. \\
In the beginning, we present the basic theory and techniques of Deep Learning, with special focus on "Residual Neural Networks" (ResNets) architectures and "Contrastive Learning" training approach. \\
Next, we explain the concept of the Internet of Things (IoT) and its cybersecurity challenges, with a particular focus on static malware analysis using Radare2. \\
Finally, we begin showing the development of the Android and Linux IoT malware classifiers, which are almost entirely performed, via Secure Shell (SSH), on a remote Linux server using the Docker platform. The codebase is written in Python using specialized libraries and frameworks for AI model development such as PyTorch and CUDA. \\
Lastly, we show some ideas for improvements and future works. 