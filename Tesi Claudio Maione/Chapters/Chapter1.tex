\setcounter{chapter}{0}

\chapter{Preliminaries}
\thispagestyle{chapterfancy}
\pagenumbering{arabic}

\section{General Definitions}

\subsection{Kullback-Leibler Divergence}\label{ch:KLD}
The relative entropy or Kullback-Leibler divergence between two probability distributions defined on the same sample space $\mathcal{X}$ is defined  \cite{mackay2003information} as:

\begin{equation}
    D_{KL}(P \| Q) = \sum_{x\in\mathcal{X}} P(x) \log\left(\frac{P(x)}{Q(x)}\right)
\end{equation}

\noindent For distributions P and Q of a continuous random variable, relative entropy is defined \cite{bishop2006pattern} to be the integral:

\begin{equation}
    D_{KL}(P \| Q) = \int_{-\infty}^{+\infty} p(z) \log\left(\frac{p(z)}{q(z)}\right)dz
\end{equation}

\noindent where p and q denote the probability densities of P and Q.

\subsection{Dirac Delta Function}\label{ch:Dirac}
Paul Dirac, in an effort to create the mathematical tools for the development of quantum field theory \cite{dirac1981principles}, introduced the Dirac delta function ($\delta$-function), a quantity depending on a parameter $x$ satisfying the conditions

\begin{equation}
    \begin{aligned} 
    & \int_{-\infty}^{+\infty} \delta(x) d x=1 \\
    & \delta(x)=0 \quad \text{for}\quad x \neq 0
    \end{aligned}
\end{equation}

\noindent $\delta(x)$ is not a function of $x$ according to the usual mathematical definition of a function, which requires a function to have a definite value for each point in its domain, but is something more general, which is called an "improper function". \\
The most important property of $\delta(x)$ is that

\begin{equation}
    \int_{-\infty}^{+\infty} f(x)\delta(x-a) d x=f(a)
\end{equation}

\noindent with $a\in\mathbb{R}$.

\subsection{Softmax}\label{ch:Softmax}
The softmax function converts a vector of $K$ real numbers into a probability distribution of $K$ possible outcomes.

\begin{equation}
    \begin{aligned} 
    & \sigma: \mathbb{R}^{K} \rightarrow\left\{z \in \mathbb{R}^{K} \mid z_{i}>0, \sum_{i=1}^{K} z_{i=1}\right\} \\ 
    & \sigma(z)_{j}=\frac{e^{z_{j}}}{\sum_{i=1}^{K} e^{z_{i}}} \quad \text{for}\quad j=1 \ldots K
    \end{aligned}
\end{equation}

\noindent It is often used as the last activation function of a neural network to normalize the output of a network to a probability distribution over predicted output classes.

\subsection{Normal Distribution}
Given two parameters $\mu,\sigma \in \mathbb{R}$, with $\sigma>0$, a normal distribution, or Gaussian distribution, is a type of continuous probability distribution for a real-valued random variable.

\begin{equation}
    f(x)=\frac{1}{\sigma \sqrt{2 \pi}} e^{-\frac{1}{2}\left(\frac{x-\mu}{\sigma}\right)^{2}}
\end{equation}

\noindent $\mu$ is the mean, or expectation, of the distribution, while the parameter $\sigma$ is its standard deviation and $\sigma^2$ is the variance.

\section{Machine Learning}
Machine learning is a subfield of artificial intelligence (AI) that focuses on the development of algorithms and models that enable computers to learn and make predictions or decisions without being explicitly programmed. The fundamental idea behind machine learning is to give computers the ability to automatically learn from data and improve their performance over time. \\
In traditional programming, developers write explicit instructions for a computer to perform a task. In contrast, machine learning systems learn from examples or data to generalize patterns and make predictions or decisions without being explicitly programmed for a specific task.

\section{Internet of Things}
The Internet of Things (IoT) refers to a network of interconnected physical devices that communicate and exchange data with each other over the internet. These devices are embedded with sensors, software, and other technologies that enable them to collect and exchange data. The goal of IoT is to create a smart, interconnected system where devices can interact and make intelligent decisions without human intervention. \\
IoT devices can be found in various industries and everyday objects, such as smart home devices, industrial machinery, wearable fitness trackers, healthcare equipment, and more. The data generated by these devices can be collected, analyzed, and used to improve efficiency, make informed decisions, and enhance user experiences.

\section{MD5, SHA-1 and SHA-256}\label{ch:SHA}
Message-Digest Algorithm 5 (MD5), Secure Hash Algorithm 1 (SHA-1) and Secure Hash Algorithm 256 (SHA-256) are cryptographic hash functions, \textit{i.e.} non-invertible functions that map a string of arbitrary length to a string of predefined length. They are required to have the following properties:

\begin{itemize}
    \item \textbf{Preimage resistance}: It should be computationally infeasible to find an input string that produces a hash equal to a given hash.
    \item \textbf{Second preimage resistance}: It should be computationally infeasible to find an input string that produces a hash equal to that of a given string.
    \item \textbf{Collision resistance}: It should be computationally infeasible to find a pair of input strings that produce the same hash.
\end{itemize}

\noindent These hash functions are commonly used in various security applications and protocols to ensure data integrity and create digital signatures.

\begin{enumerate}
    \item \textbf{Message-Digest Algorithm 5}: MD5 produces a 128-bit (16-byte) hash value, typically expressed as a 32-character hexadecimal number. \\
    Widely used in the past for integrity checking and checksums, but it is now considered weak in terms of collision resistance, and its use in security-sensitive applications is not recommended.
    
    \item \textbf{Secure Hash Algorithm 1}: SHA-1 produces a 160-bit (20-byte) hash value, typically expressed as a 40-character hexadecimal number. \\
    Like MD5, SHA-1 has been found to have vulnerabilities to collision attacks, where two different inputs can produce the same hash value. Due to these vulnerabilities, SHA-1 is no longer considered secure for cryptographic purposes.
    
    \item \textbf{Secure Hash Algorithm 256}: SHA-256 produces a 256-bit (32-byte) hash value, typically expressed as a 64-character hexadecimal number. \\
    It is widely used and considered secure for cryptographic purposes. It is commonly used in digital signatures, certificate generation, and various security protocols.
\end{enumerate}